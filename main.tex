\documentclass[12pt]{article}
\usepackage[utf8]{inputenc}
\usepackage{graphicx}
\usepackage[hidelinks]{hyperref}
\usepackage{xcolor}
\usepackage{float}
\usepackage{longtable}
\usepackage{svg}
\usepackage{tabularx}
\usepackage[a4paper, total={6in, 8in}]{geometry}

\title{
{\includegraphics[width=4.5cm, height=3cm]{images/uwlogo.png}
\includegraphics[width=3cm, height=2.5cm]{images/walogo.png}}
\\
{Software Requirements Specification | Year 4}
}
\author{E. Wiedemeier, P. Chen, S. Dabbara, S. Ashokkumar, \\ K. Sha,  K. Pachipala, R. Anandpara \\ \\ \textbf{University of Wisconsin-Madison}}
\date{November 2024}

\begin{document}

\newpage
\maketitle
\tableofcontents

\newpage
\maketitle

\section{Introduction}
\subsection{Scope of these requirements}
The WAutoDrive software stack for Year 4 is designed to handle the challenges here. We will discuss them separately.

\subsubsection{MathWorks Simulation Challenge}
The team designed scenarios to showcase the functions specified in the MathWorks Simulation Challenge scoring rubric, as well as to perform tasks for the control stack of the vehicle for the Dynamic Challenges.

\underline{Inputs:}
\begin{itemize}
    \item Ground-truth lane marking data
    \item Ground-truth object data
    \item Ground truth vehicle state
    \item Start point and end point
\end{itemize}
\underline{Outputs:}
\begin{itemize}
    \item High-level route
    \item Target poses
    \item Steering commands
    \item Torque commands
    \item Braking commands
\end{itemize}

\subsubsection{HMI Challenge}
HMI challenge details here 

\underline{Inputs:}
\begin{itemize}
    \item Item 1
\end{itemize}
\underline{Outputs:}
\begin{itemize}
    \item Item 1
\end{itemize}

\subsubsection{V2X Challenge}
V2X challenge details here

\underline{Inputs:}
\begin{itemize}
    \item Item 1
\end{itemize}
\underline{Outputs:}
\begin{itemize}
    \item Item 1
\end{itemize}

\subsubsection{Design Your Own Challenge}
\underline{Inputs:}
\begin{itemize}
    \item Camera data
    \item LiDAR data
    \item IMU data
    \item GPS data
    \item HD map data
\end{itemize}
\underline{Outputs:}
\begin{itemize}
    \item Each speed limit sign pose and number
    \item Each stop sign pose
    \item Each traffic signal pose and state
    \item Limit/lane lines pose and type
    \item Railroad crossings
    \item Correct control outputs
\end{itemize}

\subsubsection{Localization Challenge}
Localization Challenge description here 
\underline{Inputs:}
\begin{itemize}
    \item Camera data
    \item LiDAR data
    \item IMU data
    \item GPS data
    \item HD map data
\end{itemize}
\underline{Outputs:}
\begin{itemize}
    \item Each speed limit sign pose and number
    \item Each roadway object's pose and twist
    \begin{itemize}
        \item Traffic barrels/barricade
        \item Vehicles
    \end{itemize}
    \item Limit/lane lines pose and type
    \begin{itemize}
        \item Lanes formed by barrels
    \end{itemize}
    \item Railroad crossings
    \item Correct control outputs
\end{itemize}

\subsubsection{99\% Buy Off Challenge}
99\% Buy off challenge details here 

\underline{Inputs:}
\begin{itemize}
    \item Item 1
\end{itemize}
\underline{Outputs:}
\begin{itemize}
    \item Item 1
\end{itemize}

% //////////////////////////////////////////////////////////////

\section{Assumptions \& Dependencies}
\begin{itemize}
    \item[AD\_1] The team assumes weather is non-invasive. We define non-invasive as: Not actively snowing, heavily raining, foggy and not excessively dark (visibility shall be more than one mile.)
    \item[AD\_2] It is assumed that the vehicle software stack will be used during daylight hours.
    \item[AD\_3] Unless directly specified by the rules, it is assumed a reliable GPS signal is available at all times, where the environment around the car/cart does not play a role in reducing or diminishing the signal.
    \item[AD\_4] It is assumed that the test facility for the competition shall be MCity and all features are present in the documentation provided by the AutoDrive Challenge.
    \item[AD\_5] All units used by the team, where the rules allow it, will be in SI units.
    \item[AD\_6] It is assumed that the required field of view in use for distance reporting is within 90 degrees, which is the field of view of the Cepton Vista X90 provided for the competition.
\end{itemize}

% //////////////////////////////////////////////////////////////

\section{System Features and Requirements}

\subsection{Functional Requirements}

\subsubsection{Vehicle Dynamics}
\begin{itemize}
        \item[VD\_1] The vehicle (or simulated vehicle) shall receive CAN control (steering, braking, torque) messages at the required bus baud rate from the control stack.
        \item[VD\_2] The vehicle state estimator shall update at a rate of 100 Hz.
        \item[VD\_3] Vehicle maximum speed shall be 25 mph (11.2 m/s).
        \item[VD\_4] Vehicle maximum steering angle shall be $\pm$ 45 degrees.
        \item[VD\_5] Vehicle maximum lateral acceleration shall be $\pm$ 3.5 m/s.
        \item[VD\_6] Vehicle maximum longitudinal acceleration shall be $\pm$ 4 m/s.
        \item[VD\_7] Vehicle dimensions (box) are 169.5”L x 69.7”W x 63.6”H (4.306m L x 1.77m W x 1.616m H).
        \item[VD\_8] The vehicle model's geometric center shall be located below the center of the rear axle (at ground level) for the purposes of model-based control and planning.
    \end{itemize}
\subsubsection{Traffic Light, Sign and Lane Detection (TSLD)}
\begin{itemize}
    \item[TSLD\_1] The software shall detect all visible traffic lights within 40 meters and a 120° cone in front of the vehicle, regardless of traffic light orientation and type.
    \begin{itemize}
        \item[TSLD\_1\_1] The software shall quickly and correctly determine the state of each detected traffic light, including flashing lights and including following a color transition.
        \item[TSLD\_1\_2] The position of all traffic lights shall be determined with an accuracy of ± 1 meters laterally, longitudinally, and vertically (height above the ground).
        \item[TSLD\_1\_3] The most relevant traffic light shall be determined based on the Ego-Lane.
    \end{itemize}
    \item[TSLD\_2] The software shall detect all signage visible to the car within 40 meters and inside a 120° cone in front of the vehicle. 
    \begin{itemize}
        \item[TSLD\_2\_1] The sign type shall be determined. The sign types that shall be identified are stop, yield, speed limit, railroad crossing, left turn only, and right turn only.
        \item[TSLD\_2\_2] Signs determined to not be one of the types listed in TSLD\_2\_1 shall be labeled "unknown".
        \item[TSLD\_2\_3] The distance to the sign shall also be determined, with an accuracy of ± 1 meters laterally and longitudinally.
        \item[TSLD\_2\_4] For speed limit signs, the software shall determine the most relevant speed limit sign and extract the current speed limit.
    \end{itemize}
    \item[TSLD\_3] The software shall detect all lane lines and limit lines in the roadway around the vehicle.
    \begin{itemize}
        \item[TSLD\_3\_1] All visible lane lines on the same side of the road as the vehicle shall be detected.
        \item[TSLD\_3\_2] The horizontal distance to detected lane lines shall be determined with an accuracy of ± 0.1 meters.
        \item[TSLD\_3\_3] The lane line color (white or yellow) and type (dashed or solid) shall also be determined.
        \item[TSLD\_3\_4] All limit lines within 10 meters of the front of the vehicle shall be identified.
        \item[TSLD\_3\_5] The distance to limit lines shall be determined with an accuracy of ± 1 meters laterally and ± 0.5 meters longitudinally.
        \item[TSLD\_3\_6] The sign associated with the detected limit line shall be determined, when applicable.
    \end{itemize}
\end{itemize}
\subsubsection{Object Detection \& Tracking (ODT)}
\begin{itemize}
    \item[ODT\_1] The software shall detect static objects in the road within 40 meters and inside a 120° cone in front of the vehicle.
    \begin{itemize}
        \item[ODT\_1\_1] The type of static object shall be determined. The static object types that shall be identified are traffic barrels, traffic barricades, and railroad gates.
        \item[ODT\_1\_2] Static objects determined to not be one of the types listed in ODT\_1\_1 shall be labeled "unknown", but shall still be mapped and avoided.
        \item[ODT\_1\_3] The positions of detected static objects shall be determined with an accuracy of ± 1 meters.
        \item[ODT\_1\_4] The widths and heights of detected static objects shall be determined with an accuracy of ± 0.2 meters.
    \end{itemize}
    \item[ODT\_2] The software shall detect dynamic objects in the road within 40 meters and inside a 120° cone in front of the vehicle.
    \begin{itemize}
        \item[ODT\_2\_1] The type of dynamic object shall be determined. The dynamic object types that shall be identified are pedestrians, vehicles, and deer.
        \item[ODT\_2\_2] Dynamic objects determined to not be one of the types listed in ODT\_2\_1 shall be labeled "unknown", but still shall be mapped and avoided.
        \item[ODT\_2\_3] The position of detected vehicles shall be determined with an accuracy of ± 1 meters laterally and longitudinally.
        \item[ODT\_2\_4] The position of all other detected dynamic objects shall be determined with an accuracy of ± 1 meter laterally and longitudinally.
        \item[ODT\_2\_5] The width and height, and depth (when applicable) of detected dynamic objects shall be determined with an accuracy of ± 0.2 meters.
        \item[ODT\_2\_6] The relative velocity and real velocity of detected dynamic objects shall be determined with an accuracy of ± 1 m/s.
        \item[ODT\_2\_7] The direction of travel of detected dynamic objects shall be determined with an accuracy of ± 30° relative to true north.
    \end{itemize}
    \item[ODT\_3] The software shall determine which lanes (if any) are blocked by objects in the road.
\end{itemize}
\subsubsection{Positioning \& Route Planning}
\subsubsection{State Estimation (SE)}
    \begin{itemize}
        \item[SE\_1] The ego position and heading of the vehicle shall be estimated to an accuracy of 0.5 m and 1 degree.
        \item[SE\_2] The ego velocity of the vehicle shall be estimated to an accuracy to 0.5 m/s.
        \item[SE\_3] The state $\vec{x}$ of the vehicle model shall include: The $x$ and $y$ coordinates of the vehicle's geometric center in the world frame, the speed $v$ of the vehicle, and the heading angle $\psi$ from the world frame's $x$-axis, (CCW positive).
        \item[SE\_4] The input $\vec{u}$ of the vehicle model shall be the vehicle linear acceleration $\dot{v}$ and the heading angle rate $\dot{\psi}$, which is equivalent (up to a simple transformation) to the steering angle.
        \item[SE\_5] An Extended Kalman Filter (EKF) shall predict and update the state estimate as new sensor data becomes available. At a minimum, the EKF shall fuse IMU and GPS data. Other sensor information such as wheel encoders shall be fused if available.
    \end{itemize}
\subsubsection{Tiered Motion Planning (MP)}
    \begin{itemize}
        \item[MP\_1] User input shall be a GPS coordinate to reach.
        \item[MP\_2] The high-level planner shall be map-based. The HD map of M-City shall be transformed into a directed graph via adjacency lists. The graph nodes closest to the specified start and goal GPS points and headings by the user shall be selected as the start and goal nodes. The A* search algorithm shall compute the sequence of nodes, and correspondingly, the sequence of lanes to reach the goal. This sequence shall be passed to the mid-level planner.
        \begin{itemize}
            \item[MP\_2\_1] The directed graph shall encapsulate each HD map road segment into a node.
            \item[MP\_2\_2] Nodes shall contain metadata about its corresponding road segment (i.e. number of lanes, target lane (filled during path planning), waypoints within the lanes, speed limit, heading, etc.).
            \item[MP\_2\_3] The output sequence of lanes shall be a legal sequence to traverse.
        \end{itemize}
        \item[MP\_3] The mid-level planner shall be perception-based. Using data from the perception stack (object, lane, traffic sign and traffic light information), combined with current state, the planner shall make a decision on what action to take next (lane change, lane keep, stop, wait, etc).
        \begin{itemize}
            \item[MP\_3\_1] If the current lane is blocked by a static obstacle but there is a open (legal to change into) lane to the right or the left, and there remains greater than 10 seconds at the current velocity to switch lanes and switch back to the original route, the car shall change lanes immediately while signaling with blinker lights.
            \item[MP\_3\_2] If there are no open lanes in front of the vehicle, or not enough time to switch lanes and switch back to the target lane, preventing the vehicle from continuing on the planned route, the car shall stop for 5 seconds, perform rerouting using the high-level planner with the updated road information, and starting following the new planned route to the destination. If all lanes are obstructed and the vehicle cannot proceed, the planner shall cause the vehicle to remain at a complete stop.
            \item [MP\_3\_3] The planner shall cause the vehicle to come to a complete stop at least 0.5m before the edge of the limit line when approaching a stop sign or red traffic light.
            \item [MP\_3\_4] If dynamic obstacles are encountered, the planner shall cause the vehicle to come to a complete stop 10 meters away from the obstacle until the dynamic obstacles have left the road.
        \end{itemize}
        \item[MP\_4] The low-level planner shall be physics-based. It shall use the keyframes from the mid-level planner to generate a smooth, kinodynamically feasible trajectory for the vehicle model. This process involves solving an optimal control problem that is relaxed into three second-order cone programs (SOCPs) that are sequentially solved using YALMIP with MOSEK at a rate of at least 10 Hz. These SOCPs find the safe path, trajectory duration, and speed profile, respectively. By solving these SOCPs, a sub-optimal but feasible trajectory that takes into account the current state and target state of the vehicle shall be generated.
        
        \item[MP\_5] The controller shall be physics-based. It shall use a PID controller to apply the trajectory inputs generated by the low-level planner, and output throttle, brake, and steering values for the car.
    \end{itemize}
\subsubsection{AV Mode Blue Light}
\begin{itemize}
    \item[AVL\_1] A set of AV Mode Blue Lights shall be mounted on the exterior of the vehicle or cart. These lights warn others around the vehicle that Autonomous Vehicle Mode has been activated. The light shall receive constant communication from the compute stack through CAN.
    \begin{itemize}
        \item[AVL\_1\_1] When the vehicle is in Autonomous Vehicle Mode, the light shall be turned on.
        \item[AVL\_1\_2] When the vehicle is turned off or is operating in Manual Mode, the light shall be off.
        \item[AVL\_1\_3] In the case where CAN communication is lost, the light shall flash at a frequency of 1Hz.
    \end{itemize}
    \item[AVL\_2] The light shall be visible from at least 200 ft on all sides. Mounting solutions are determined by the team.
    \item[AVL\_3] A second AV Mode Light shall be mounted on the interior of the vehicle on the dashboard. This light serves as an indicator for passengers that the autonomous control system is currently in control. The light shall be visible by all occupants and shall be in sync with the exterior AV Mode Light.
    \item[AVL\_4] The AV Mode Light Controller will send a CAN status message detailing the state of the light at a frequency of 10 Hz.
\end{itemize}
\subsubsection{V2X}
\begin{itemize}
    \item[V2X\_1] The V2X system shall decode a PCAP file using an ASN1 executable, ensuring the proper transmission of V2X messages.
    \item[V2X\_2] The system shall provide a fully functional Scoring CAN message to communicate V2X information.  
    \item[V2X\_3] The V2X system shall transmit information in scenarios where perception may be impaired, such as in situations with poor visibility.
    \item[V2X\_4] The V2X system shall transmit traffic light information along with data from signboards.
\end{itemize}



% //////////////////////////////////////////////////////////////

\subsection{Non-Competition Requirements}
\begin{itemize}
    \item[NCR\_1] The Perception Cart must be collapsible and packed into a shipping container in 30 minutes. This includes unmounting and storing the sensors and their fixtures inside the cart for shipping.
    \item[NCR\_2] The control stack for the MathWorks Simulation Challenge shall be compatible with the computing hardware on the Perception Cart and the Chevy Bolt. As a result of this, the control stack will be usable in upcoming competition years.
    \item[NCR\_3] The Perception Cart must run for at least 2 hours on battery power. The Perception Cart must be self contained, meaning you can edit and run code without the need for external power, cords (e.g. ethernet), and must have a monitor.
    \item[NCR\_4] The sensor mounts should be easily transferable between the Perception Cart and the Chevy Bolt. This will allow for minimal time loss and easier quick testing on the Perception Cart when necessary.
    \item[NCR\_5] The Chevy Bolt shall be equipped with two additional internal monitors. One monitor shall be mounted at the front passenger seat. The second monitor shall be mounted at the rear seat behind the driver. Both must allow for a person to simultaneously be seated. This will allow for three people to sit in the vehicle, in addition to the safety driver, with all three passengers being able to contribute to testing.
    \item[NCR\_6] The Intel server shall be set up such that members will be able to remote into it (SSH, VNC).
\end{itemize}

% //////////////////////////////////////////////////////////////

\subsection{External Interface Requirements}
\begin{itemize}
    \item[EIR\_1] User Interfaces: For development and debugging, we will interface with the server through IPMI. For competition, we will interface with a dedicated Human-Machine Interface (HMI) per per competition guidelines
    \item[EIR\_2] Computing Hardware Interfaces: The software stack shall run on an Intel-provided server. The computer shall function as a Linux box, and be accessible remotely via Secure Shell (SSH). Available ports should consist of at least four independent USB ports, a video output port, an ethernet port, and be WiFi enabled.
    \item[EIR\_3] Mounting Hardware Interfaces: The mounts for our sensors and Autonomous Blue Light shall not interfere with scoring equipment provided by competition scorers.
    \item[EIR\_4] Out-going Communication Interfaces: The CAN Bus Wiring shall contain at least one standard OBD-II port that is compatible with off-the-shelf debugging hardware which allows attachment of competition scoring hardware and allow the team to analyze and capture CAN Bus traffic with NEO VI Interface Box.
    \item[EIR\_5] Inter-Module Communication Interfaces: All the hardware designed by the team(e.g. HMI, ABL) will use a 4-pin locking connector that connects to 4 custom CAN Bus Splitters. Each Splitter hosts one of four CAN Bus Systems (HS,CE,LS,SC), the splitter shall have extra ports to allow for future hardwares.
    \end{itemize}

% //////////////////////////////////////////////////////////////

\subsection{Conclusion}
The WAutoDrive software stack for Year 3 provides a comprehensive solution for handling multiple dynamic driving challenges. The Intersection Challenge, Construction Challenge, and MathWorks Simulation Challenge are each addressed with specific functionalities, ensuring robust and reliable operation in diverse scenarios.

Assumptions such as clear weather, daylight driving, and consistent GPS signals allow the software stack to focus on functional requirements. These include vehicle dynamics, traffic light and sign detection, and object tracking, ensuring safe and efficient navigation through various environments.

The stack's positioning and route planning components implement state estimation, tiered motion planning, and route adjustments in real time. This enables seamless navigation through different driving conditions and obstacles. Safety features such as AV Mode Blue Light and V2X functionality contribute to reliable autonomous vehicle operation.

Non-competition requirements, including portability of the perception cart, software compatibility, and collapsibility, ensure effective testing and evaluation of the stack's components. External interface requirements guarantee seamless integration of the software stack with both hardware and scoring mechanisms, completing the WAutoDrive stack for competition and beyond. 

\section{Appendix 1: Test Results}
In this appendix, your team will include the test results for each and every requirement in your SRS, in
table format. This table is required to include the following columns:
a) SRS Requirement Identifier – list the traceable requirement identifier. Requirement itself does
not need restated. (Draft and Final Required)
b) Test Method – did you test this requirement on vehicle? On a test bench? Via simulation?
(Draft and Final Required)
c) Pass or Fail or Unable to Test (Final Required)
d) Comments with respect to failure – why did the test case fail? Does the requirement need
updated? Is it OK to proceed with this failure? Was there a test setup issue? (Final Required)
Finally, please include a paragraph summarizing your testing. Were there any surprises? Did you need
to update any requirements to be testable? Add more requirements for coverage? What test method
was the most valuable?

\begin{table}[h!]
\centering
\begin{tabularx}{\textwidth}{|X|X|X|X|}
\hline
Identifier & Test Method & Pass/Fail & Comments \\ \hline
Data 1   & Data 2   & Data 3   & Data 4   \\ \hline
Data 5   & Data 6   & Data 7   & Data 8   \\ \hline
Data 9   & Data 10  & Data 11  & Data 12  \\ \hline
\end{tabularx}
\caption{Requirements test results}
\label{tab:four-column-table}
\end{table}

\section{Appendix 2: Open Source Software Disclosure Form}
In this appendix, you will attach your team's open-source software disclosure form using the provided
template. Please provide a detailed list of all the software packages/bundles that you leverage in your
software system but did not develop. This should include, but is not limited to, the operating system,
ROS/dSpace packages used directly, and any code used to run your vehicle software system that was
not developed by your team.
Please ensure that your submission follows the OSS disclosure template Excel sheet and that all
columns are displayed on a single page.


\end{document}
