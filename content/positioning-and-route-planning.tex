\subsubsection{Positioning \& Route Planning}
\subsubsection{State Estimation (SE)}
    \begin{itemize}
        \item[SE\_1] The ego position and heading of the vehicle shall be estimated to an accuracy of 0.5 m and 1 degree.
        \item[SE\_2] The ego velocity of the vehicle shall be estimated to an accuracy to 0.5 m/s.
        \item[SE\_3] The state $\vec{x}$ of the vehicle model shall include: The $x$ and $y$ coordinates of the vehicle's geometric center in the world frame, the speed $v$ of the vehicle, and the heading angle $\psi$ from the world frame's $x$-axis, (CCW positive).
        \item[SE\_4] The input $\vec{u}$ of the vehicle model shall be the vehicle linear acceleration $\dot{v}$ and the heading angle rate $\dot{\psi}$, which is equivalent (up to a simple transformation) to the steering angle.
        \item[SE\_5] An Extended Kalman Filter (EKF) shall predict and update the state estimate as new sensor data becomes available. At a minimum, the EKF shall fuse IMU and GPS data. Other sensor information such as wheel encoders shall be fused if available.
    \end{itemize}
\subsubsection{Tiered Motion Planning (MP)}
    \begin{itemize}
        \item[MP\_1] User input shall be a GPS coordinate to reach.
        \item[MP\_2] The high-level planner shall be map-based. The HD map of M-City shall be transformed into a directed graph via adjacency lists. The graph nodes closest to the specified start and goal GPS points and headings by the user shall be selected as the start and goal nodes. The A* search algorithm shall compute the sequence of nodes, and correspondingly, the sequence of lanes to reach the goal. This sequence shall be passed to the mid-level planner.
        \begin{itemize}
            \item[MP\_2\_1] The directed graph shall encapsulate each HD map road segment into a node.
            \item[MP\_2\_2] Nodes shall contain metadata about its corresponding road segment (i.e. number of lanes, target lane (filled during path planning), waypoints within the lanes, speed limit, heading, etc.).
            \item[MP\_2\_3] The output sequence of lanes shall be a legal sequence to traverse.
        \end{itemize}
        \item[MP\_3] The mid-level planner shall be perception-based. Using data from the perception stack (object, lane, traffic sign and traffic light information), combined with current state, the planner shall make a decision on what action to take next (lane change, lane keep, stop, wait, etc).
        \begin{itemize}
            \item[MP\_3\_1] If the current lane is blocked by a static obstacle but there is a open (legal to change into) lane to the right or the left, and there remains greater than 10 seconds at the current velocity to switch lanes and switch back to the original route, the car shall change lanes immediately while signaling with blinker lights.
            \item[MP\_3\_2] If there are no open lanes in front of the vehicle, or not enough time to switch lanes and switch back to the target lane, preventing the vehicle from continuing on the planned route, the car shall stop for 5 seconds, perform rerouting using the high-level planner with the updated road information, and starting following the new planned route to the destination. If all lanes are obstructed and the vehicle cannot proceed, the planner shall cause the vehicle to remain at a complete stop.
            \item [MP\_3\_3] The planner shall cause the vehicle to come to a complete stop at least 0.5m before the edge of the limit line when approaching a stop sign or red traffic light.
            \item [MP\_3\_4] If dynamic obstacles are encountered, the planner shall cause the vehicle to come to a complete stop 10 meters away from the obstacle until the dynamic obstacles have left the road.
        \end{itemize}
        \item[MP\_4] The low-level planner shall be physics-based. It shall use the keyframes from the mid-level planner to generate a smooth, kinodynamically feasible trajectory for the vehicle model. This process involves solving an optimal control problem that is relaxed into three second-order cone programs (SOCPs) that are sequentially solved using YALMIP with MOSEK at a rate of at least 10 Hz. These SOCPs find the safe path, trajectory duration, and speed profile, respectively. By solving these SOCPs, a sub-optimal but feasible trajectory that takes into account the current state and target state of the vehicle shall be generated.
        
        \item[MP\_5] The controller shall be physics-based. It shall use a PID controller to apply the trajectory inputs generated by the low-level planner, and output throttle, brake, and steering values for the car.
    \end{itemize}